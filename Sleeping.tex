\documentclass{article}
\usepackage{ctex}
\usepackage{multicol}
\usepackage[top=1in, bottom=1in, left=1.25in, right=1.25in]{geometry}
\usepackage{lscape}
\author{Fangfang Li}
\date{April 19, 2018}
\title{6 surprising things you might be doing while asleep}
\begin{document}
\maketitle
One third of our lives are spent in sleep. Sleep is so important, but we know so little about sleep that even scientists can't fully crack the sleep code. When you sleep, you probably don't know, your body is still busy!
So, what are you doing after you fall asleep? Revelation: what does your body do when you sleep?

The reason for reading this article is because I really want to know what people's bodies are doing when they sleep. The article introduces six things that people can do. You might be asleep but your hypothalamus is not. It's carefully keeping time for you as part of your circadian rhythm. This not only helps you feel tired so that you go to sleep with the release of melatonin, but a protein called PER is released in the morning that gradually wakes you up, often right before your alarm clock is set to go off. About 5 percent of adults do it (it's slightly more common in children), and it can happen during any stage of sleep. Most people grind their teeth while they are asleep, at least sometimes. This habit, called bruxism, can be caused by emotional or psychological states like stress or anxiety, from an abnormal bite (misalignment of your teeth), or even from sleep apnea. While you are asleep your brain clears out some memories and cements and reorganizes others. Your muscles are frozen for part of every night. Your body loses weight at night when you sleep. Each night you lose about a pound due to the water vapor you expel while breathing.

\end{document}
