\documentclass{article}
\usepackage{ctex}
\usepackage{multicol}
\usepackage[top=1in, bottom=1in, left=1.25in, right=1.25in]{geometry}
\usepackage{lscape}
\author{Fangfang Li}
\date{April 21, 2018}
\title{Eight healthy signs}
\begin{document}
\maketitle
8 signs that you're actually healthy �� even if it doesn't feel like it
You don't have to run a marathon, do all the yoga, to be healthy.
Here are eight signs of health.

You eat when you're hungry and stop when you're full. This sentence looks like a guff�� but it's a sign of good health. You're eating a varied diet rich in whole foods. You're eating enough. Don't be afraid to be fat by eating too much. This simple behavior is a hallmark of healthy eating. Remember, calories aren't your enemy or some evil force to be reduced at all costs. They're an energy source that helps you live your life and do what you love. And if you're not eating enough of them, you could end up feeling moody, weak, achy, and more. You can make it up two flights of stairs and feel pretty good. The recommended amount of exercise for good health is 150 weekly minutes of moderate activity, like brisk walking. You embrace your full range of emotions. I think emotional wellbeing is embracing the whole gamut of emotions and understanding that they're all normal. Same goes for dealing with stress. You can wake up without an alarm clock. It should come as no surprise that selling yourself short on sleep is terrible for your body and brain. Research has linked insufficient sleep to increased risk of Alzheimer's, obesity, stroke, and diabetes. That's why it's important to shoot for eight hours a night. You're not falling asleep too quickly. Falling asleep in an average amount of time �� roughly 10 to 20 minutes �� is a sign that your sleep is pretty good. You have the energy to do the things you want to do.


No matter what, remember that health is individualized. If you're worried you're not healthy �� or you're just curious about where you stand �� make sure to see your doctor.
\end{document}