\documentclass{article}
\usepackage{ctex}
\usepackage{multicol}
\usepackage[top=1in, bottom=1in, left=1.25in, right=1.25in]{geometry}
\usepackage{lscape}
\usepackage{graphicx}
\usepackage{subfigure}
\usepackage{cite}
\bibliographystyle{plain}
\author{Fangfang Li}
\date{May,1,2018}
\title{Peppa Pig}
\begin{document}
\maketitle
Since last summer, the young pig has become the topic of countless memes, jokes, and short videos — especially since she was jokingly cast as a shehuiren, a slang term for a gangster.
%插入图片
\begin{figure}[htbp]
%\small
\centering
\includegraphics[width=0.5\textwidth]{P.jpg}
\caption{Peppa Pig}
\label{1}
\end{figure}
\par On viral short video platform Douyin, more than 30,000 videos appear under the Peppa Pig hashtag. The series is rated an astronomical 9.2 on review platform Douban, and its fifth season, released last October, has been viewed over 14 billion times on video platform Youku. Peppa Pig theme parks are even slated to open in Beijing and Shanghai next year — just in time for the Year of the Pig in 2019. Unofficial, fan-made Peppa Pig chat stickers and emojis have proliferated in the last year, and a Weibo social media account that shares such stickers now has over 200,000 fans — nearly four times the followers of the famous pig’s official account. In November, a fan-made Peppa Pig video dubbed in Chongqing dialect became an instant hit on video-sharing platform Bilibili, leading it to becoming dubbed in more than 15 other Chinese tongues, including Shanghainese, Sichuanese, and the northeastern accent. Watches and bracelets with various Peppa Pig designs have sold over 100,000 units on Taobao in the last month.
%\bibliographystyle{plain}
\footnote{from"China Daily"}
\begin{thebibliography}{99}
\bibitem{pa}Jane Raymond.~Interactions of attention,~emotion and motivation~[J]~.Progress in Brain Research.~2009~(3):~293~-~308.
\end{thebibliography}
\end{document}